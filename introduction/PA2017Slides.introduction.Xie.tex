\documentclass[handout]{beamer}
\usepackage{lmodern}

\usepackage{ctex}

\usetheme{CambridgeUS} % try Madrid, Pittsburgh
\usecolortheme{beaver}
\usefonttheme[onlymath]{serif} % try "professionalfonts"


\graphicspath{{./figure/}}

\setbeamertemplate{itemize items}[default]
\setbeamertemplate{enumerate items}[default]

\usepackage{amsmath, amsfonts, latexsym, mathtools, extarrows}
\usepackage{centernot} % for ``not implies'' symbol
\DeclareMathOperator*{\argmin}{arg\,min}
\DeclareMathOperator*{\argmax}{arg\,max}

\usepackage{pifont}
\usepackage{hyperref}
\usepackage{comment}

\usepackage[normalem]{ulem}
\newcommand{\middlewave}[1]{\raisebox{0.5em}{\uwave{\hspace{#1}}}}

\usepackage{graphicx, subcaption}

\usepackage{verbatim}
\usepackage{algorithm}
\usepackage[noend]{algpseudocode}

% table
\usepackage{amstext} % for \text macro
\usepackage{array}   % for \newcolumntype macro
\newcolumntype{L}{>{$}l<{$}} % math-mode version of "l" column type
\newcolumntype{C}{>{$}c<{$}} % math-mode version of "c" column type
\newcolumntype{R}{>{$}r<{$}} % math-mode version of "r" column type
\newcommand{\pno}[1]{\textcolor{blue}{\scriptsize [Problem: #1]}}
\newcommand{\set}[1]{\{#1\}}
\newcommand{\cmark}{\textcolor{red}{\ding{51}}}
\newcommand{\xmark}{\textcolor{red}{\ding{55}}}
%%%%%%%%%%%%%%%%%%%%%%%%%%%%%%%%%%%%%%%%%%%%%%%%%%%%%%%%%%%%%%
% for fig without caption: #1: width/size; #2: fig file
\newcommand{\fignocaption}[2]{
	\begin{figure}[htp]
		\centering
		\includegraphics[#1]{#2}
	\end{figure}
}
% for fig with caption: #1: width/size; #2: fig file; #3: fig caption
\newcommand{\fig}[3]{
	\begin{figure}[htp]
		\centering
		\includegraphics[#1]{#2}
		\caption[labelInTOC]{#3}
	\end{figure}
}
\newcommand{\titletext}{Programming Assignment Lecture \uppercase\expandafter{\romannumeral1}}

%%%%%%%%%%%%%%%%%%%%
\title{\titletext}
\subtitle{Introduction}
\author{Xie}
\institute{xiemhemail@gmail.com}
\date{Sep ??, 2017}
\AtBeginSection[]{
	\begin{frame}[noframenumbering, plain]
		\frametitle{\titletext}
		\tableofcontents[currentsection, sectionstyle=show/shaded, subsectionstyle=show/show/hide]
	\end{frame}
}
%%%%%%%%%%
\begin{document}
	\maketitle
	
\begin{frame}{Basic Info}
	\begin{itemize}
		\item 课程内容
		\begin{itemize}
			\item PA1-PA??
		\end{itemize}
		\item 上课时间
		\begin{itemize}
			\item ???
		\end{itemize}
		\item 考核方式
			\begin{itemize}
				\item 日常按阶段提交工程 + 实验报告
				\item 请在阶段deadline前提交工程
			\end{itemize}
		\item \alert{教程网站}
		\begin{itemize}
			\item https://nju-ics.gitbooks.io/ics2017-programming-assignment/
			\item \huge 请大家每天至少关注一次页面中的“最新消息”
		\end{itemize}
	\end{itemize}
\end{frame}

\begin{frame}{Teacher and T.A.s}
	%TODO
	emmmmm
\end{frame}

\section{ICS,PA and Computer System}

\begin{frame}
	\centering{\huge Why we need learn ICS? -- yzh}
\end{frame}

\begin{frame}[fragile]{Motivation}
\begin{exampleblock}{Question}
\begin{verbatim}
int main()
{
    printf("Hello World");
    return 0;
}
\end{verbatim}
What the computer are doing when you execute the program above?
\end{exampleblock}

\begin{alertblock}{Tip}
	This may appear in exam.
\end{alertblock}

\end{frame}




\begin{frame}{System Stack}
\begin{table}
	\centering
	\begin{tabular}{|c|}
		\hline
		Application \\	
		\hline
		Algorithm\\
		\hline
		Programming Language\\
		\hline
		Operating System/Virtual Machines\\
		\hline
		\alert{Instruction Set Architecture}		\\
		\hline
		Micro-architecture\\
		\hline
		Register-Transfer Level\\
		\hline
		Gates\\
		\hline
		Circuits\\
		\hline
		Devices\\
		\hline
		Physics\\
		\hline
	\end{tabular}
\end{table}
\end{frame}



\begin{frame}{Aims of PA}
	\begin{description}
		\item[Aims]
		\begin{itemize}
		\item	\textbf{Systems thinking}
		\item	Understand how program run on a computer
		\item	Enhance \textbf{coding} ability
		\item	Prepared for later courses (OS,Compiling)
		\end{itemize}
		\pause
		\item[Way]Complete a tiny but entire computer system and run program on it.
		\pause
		\item[PA] \alert{\huge NEMU}(\textit{i.e. NJU Emulator})
	\end{description}
\end{frame}



\begin{frame}{Resources}
	\begin{description}
		\item [Plantform and tools] IA-32 + GNU/Linux + gcc + C	
		\item [Guidebook] https://nju-ics.gitbooks.io/ics2017-programming-assignment/
		\item [Skeleton]https://github.com/NJU-ProjectN/ics-pa
	\end{description}
	\begin{alertblock}{Tip}
		You can download the PDF or epub version of guide in github.
	\end{alertblock}
\end{frame}

%\begin{frame}{学生体会}
%	2017.1.11 2:14AM   
%	
%	不相信所谓的执念, 本已打算放弃, 交换了bt指令的源操作数和目的操作数, 出现了在别人家的电脑上出现过无数次的画面, 耳边是Anan Ryoko的Refrain, 却已经难以抑制地在只有一个人的寝室里失声痛哭. 
%	
%	
%	从汇编语言课程开始听闻PA实验, 举步维艰地进行每一个阶段, 无数个不眠的深夜, 第二天拖着疲惫的身体睁着朦胧的睡眼去上课, 我一次次地怀疑自己, 一次次想要放弃. 现在想来, 这也印证了”成长是一个痛苦的过程”这段文字. 如果说承认自己的软弱, 是成长的第一步. 那么可以恬不知耻地说, 我已经真正从19岁走向20岁了. 是的, 你在专业上的技不如人, 迟早有一天会找上来, 我已经真正感受到了. 既然选择了远方, 便只顾风雨兼程. …当时有很多规划, 现在看来都在不断调整, 然而有一点没有变, 我的人生职业目标: 成为一名IT精英. 仙剑的运行作为IT路上的第一个里程碑吧. 
%\end{frame}



\section{The Story of Computer}
\begin{frame}{DLC}
	\begin{enumerate}
		\item We all have completed the Digital Logic Circuit Course.(Taught by 张泽生 吴海军)
			\begin{enumerate}
				\item summator
				\item register
				\item multiplexer
				\item \textit{etc.}
			\end{enumerate}
		\item 
	\end{enumerate}
\end{frame}

\section{Brand new PA}

\begin{frame}{The End}
	\fignocaption{scale=0.8}{studentFeedback.png}
\end{frame}


\end{document}